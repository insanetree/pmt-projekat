\documentclass{beamer}

\mode<presentation>

  \usetheme{Copenhagen}
  \setbeamercovered{transparent}
%  \usecolortheme{seahorse}
  \usecolortheme{rose}
%  \usecolortheme{wolverine}
%  \useoutertheme{infolines}


\usepackage[T1]{fontenc}
\usepackage[serbian]{babel}
\usepackage[cp1250]{inputenc}
\usepackage{times}


\usepackage{graphics}
\usepackage{eurosym}
\usepackage{multirow}
\usepackage{multimedia}
\usepackage{color}
\usepackage{multirow}
%\usepackage{subfig}


\title[Simulacija zagu\v{s}enja \hspace{1em} {\scriptsize \insertframenumber\ / \inserttotalframenumber}]
{Simulacija zagu\v{s}enja pomo\'{c}u ns-3\\[3ex]
\small
projekat iz predmeta\\
Principi modernih telekomunikacija (IR3PMT)\\
2023.}

\author[A. Popovi\'{c} --- IR3PMT 2023.]{Aleksandar Popovi\'{c}}

\date[]{}


%\AtBeginSection[]
%{
%  \begin{frame}<beamer>
%    \frametitle{Sadr�aj}
%    \tableofcontents[currentsection,hideothersubsections]
%  \end{frame}
%}


%\beamerdefaultoverlayspecification{<+->}


\begin{document}

\begin{frame}
%  \vfill
  \titlepage
\end{frame}


\begin{frame}
    \frametitle{Sadr�aj}
    \tableofcontents
\end{frame}


\section{Uvod}
% nazivi odeljaka ce se prikazati u sadrzaju

\begin{frame}
 \frametitle{NS-3}

\begin{columns}
	\column{.60\textwidth}
	  \begin{itemize}
		\item <1-> Simulator mre\v{z}a
		\item  <2-> Softver otvorenog koda
		\item <3-> Skripte se pi\v{s}u u jeziku C++
		% <k-> znaci da se sadrzaj u tom redu prikazuje od k-tog klika nadalje
	\end{itemize}
	\column{.40\textwidth}
	\begin{figure}[t]
		\centering
		\includegraphics[width=\textwidth]{../slike/ns-3.png}
	\end{figure}
\end{columns}

\end{frame}

\begin{frame}
\frametitle{Matemati�ki model}
  \begin{itemize}
	\item	$T_Q$ --- trajanje �ekanja,
	\item	$T_S$ --- trajanje obrade ($=\tau$)
	\end{itemize}
\end{frame}

\begin{frame}
 \frametitle{Analiza}

Iskori��enost servera: $\rho = \lambda / \mu$\\[1ex]
Pollaczek-Khinchinova formula
 \begin{columns}[T]
	\column{.35\textwidth}
	 \begin{block}{$M$/$G$/1}<1->
	$$T_{Q}=\frac{\lambda \overline{\tau^{2}}}{2(1-\rho)}
$$
\end{block}
	\column{.55\textwidth}
  \begin{exampleblock}{$M$/$D$/1}<2->
	$$\overline{\tau^{2}}=\frac{1}{\mu^{2}} \quad \Rightarrow \quad
	T_{Q}=\frac{\rho}{2\mu(1-\rho)}
$$
\end{exampleblock}
\end{columns}
\end{frame}


\section{Prora�un \dots}

\begin{frame}
 \frametitle{Primer}
 \begin{exampleblock}{Zadatak}<1->
	Poruke fiksne du�ine 1000 B, koje se generi�u po Poissonovoj raspodeli, prenose se linijom na kojoj je protok 9600 b/s. Iskori��enost linije iznosi 70\%. Odrediti prose�no trajanje �ekanja.
\end{exampleblock}
\begin{alertblock}{Re�enje:}<2->
\begin{gather*}
\uncover<2-> \tau=\mu^{-1}=\displaystyle{\frac{1000\cdot8\mbox{ b}}{9600\mbox{ b/s}}}=0,\!83\mbox{ s} ,\quad \rho=\lambda \tau = 0,\!7\ \\[1ex]
\uncover<3->{T_{Q}=\frac{\rho}{2\mu(1-\rho)}= 0,\!97\mbox{ s}}
\end{gather*}

\end{alertblock}
\end{frame}

\end{document}

