\documentclass[a4paper, 12pt, projekat]{etf}
\usepackage[intlimits]{amsmath}
\usepackage{amsmath, amsfonts, amssymb, graphicx}
\usepackage[serbian]{babel}
\usepackage[T1]{fontenc}
\usepackage[utf8]{inputenc}
\addto\captionsserbian{%
	\renewcommand{\bibname}%
	{Literatura}%
}


\title{Simulacija zagušenja pomoću ns-3}
\author{Aleksandar Popović}
\indeks{2020/0059}
\date{jul 2023.}
\mentor{prof. dr Milan Bjelica}
\predmet{Primena modernih telekomunikacija}

\begin{document}
	\maketitle
	\begin{abstract}
		OVDE NAPISATI SA\v{Z}ETAK 
	\end{abstract}
	\begin{keywords}
		ns-3, simulacija, zagu\v{s}enje
	\end{keywords}
	\tableofcontents
	\listoffigures
	\listoftables
	\chapter{Uvod}
	U ovom radu će se obraditi simulacija zagušenja mreže pomoću simulatora NS-3(Network Simulator 3). 
	Simulator je slobodan softver; njegov izvorni kod je javno dostupan za proučavanje i izmenu.\\
	U simulaciji će se pokazati kako preopterećenje veze dovodi do gubitaka paketa i kako se performanse poboljšavaju posle povećanja kapaciteta.\\
	U nastavku će biti opisan postupak pokretanja simulacije, opis simulirane topologije i pregled rezultata.
\end{document}